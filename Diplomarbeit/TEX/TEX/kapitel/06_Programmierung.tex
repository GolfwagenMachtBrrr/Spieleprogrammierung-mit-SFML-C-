% Codedarstellung - Hinweise
% in 00_Einstellungen wird das Paket eingebunden
% in diplomarbeit.tex gibt es weitere Details dazu

\section{Struktur}
Im folgenden Abschnitt wird die Generelle Klassenstruktur vorgestellt

% Code-BEGINN
\begin{lstlisting}	
	class
	{
	 public:
		// Oefentliche Methoden
	        void Initialize(); 
		void Update(); 
		void Draw(); 
 	 private: 
		// Private Methoden
	 public: 
		// Oeffentliche Variablen
	 private:
		// Private Variablen
	};
\end{lstlisting}
\captionof{lstlisting}{Klassenstruktur}
% Code-ENDE

% Code-BEGINN
\section{Schreibstil}
Das gesamte Programm wurde mit einem einheitlichen Stil geschrieben, dieser wird in diesem Abschnitt vorgestellt. 

Methoden beginnen immer mit einem Großbuchstaben, so werden diese klar von Variabeln getrennt.

% Code-BEGINN
\begin{lstlisting}	
	class
	{
	 public:
		void Beispielsmethode(); 
	};
\end{lstlisting}
\captionof{lstlisting}{Methoden}
% Code-ENDE

Variablen welche member einer Klasse sind werden mit einem preafix gekennzeichnet und kleingeschrieben

% Code-BEGINN
\begin{lstlisting}
	class
	{
		// Methoden
		...
	 public: 
		// Oeffentliche Variablen werden mit dem Praefix p_ makiert
	 private:
		// Private Variablen werden mit dem Praefix m_ makiert
	};
\end{lstlisting}
\captionof{lstlisting}{membervariabeln}
% Code-ENDE

Lokale Variabeln zu gänze kleingeschrieben und erhalten keinen Praefix. 

% Code-BEGINN
\begin{lstlisting}
	class
	{
	 public:
		void Beispielsmethode()
		{
			int beispielsvariabel; 
		}
		...
		// Membervariabeln
	};
\end{lstlisting}
\captionof{lstlisting}{lokale variabeln}
% Code-ENDE

Bei Parametern wird der erste Buchstabe großgeschrieben, dadurch können Parameter von Member- und Lokalvariabeln unterschieden. 
 
% Code-BEGINN
\begin{lstlisting}
	class
	{
		public:
		void Beispielsmethode(int Beispielsparameter)
		{
			int beispielsvariabel = Beispielsparameter; 
		}
		...
		// Membervariabeln
	};
\end{lstlisting}
\captionof{lstlisting}{Parameter}
% Code-ENDE



% Noch nicht Fomratier  oder in richtiger Reinfolge
\section{Generieren der Karte}

Jedes Spiel braucht eine Karte, in diesem Abschnitt werden wir uns Schrittweise die Implementierung der Map Klasse anschauen. 
% Code-BEGINN

\begin{lstlisting}
class Map
{
public:
	Map()
	{}
	
	void Initialize();
	void Generate();
	void Draw();
	... 
};
\end{lstlisting}
\captionof{lstlisting}{Map}
% Code-ENDE

Die Methode "Initialize" hat den Zweck member Variablen einen Wert zuzuweisen und gegenbenfalls in der Klasse enthaltene Objekte zu Initalisieren.
\begin{lstlisting}
	...
	void Initialize();
	...
\end{lstlisting}

Die Idee ist es die Karte als 2-Demensionales Liste aus "Tiles"  darzustellen, vorab definieren wir also die Größe Karte, sowie die Größe der "Tiles". 
\begin{lstlisting}
	...
	void Initialize(const sf::Vector2u& tilesize, const int& width, const int& height);
	...
\end{lstlisting}

Diese Werte sollen in der Klasse abgespeichert werden. Daher definieren wie nun die drei Member "tilesize", "width", und "height". 
\begin{lstlisting}
	...
 	private: 
 		sf::Vector2u m_tilesize; 
 		int m_width; 
 		int m_height; 
	...
\end{lstlisting}

Anschließend weisen wir den Membern den jewiligen Parametern zu. 

\begin{lstlisting}
	...
	void Initialize(const sf::Vector2u& tilesize, const int& width, const int& height)
	{
		m_tileSize = tilesize;
		m_width = width; 
		m_height = height; 
	}
	...
\end{lstlisting}

\captionof{lstlisting}{Map::Initialize()}

Die Methode ist vorerst fertiggestellt, nächster Schritt ist nun die Generierung. Wie schon erwähnt ist die Idee, die Karte als 2-Demensionale Liste bestehend aus "Tiles" darzustellen. Ähnlich wie das auch andere Spielen machen. (Beispiel Anführen) Die einzelnen Tiles sollen Informationen über Textur, Position und deren ID beeinhalten. Hierfür verwenden wir den Datentyp "Struct". Structs unterscheiden sich in c++ nicht wesentlich von Klassen, jedoch werden wir aus Stilgründen den Typ Struct verwenden. Wichtig: Structs sollen jediglich informationen beeinhalten und üben sonst keine Funktion aus. 



\begin{lstlisting}
	struct Tile
	{
		...
		unsigned int tile_ID;
		sf::Vector2f tile_position;
		
		sf::IntRect* tile_texRect;
		sf::Sprite   tile_sprite;
		
	};
\end{lstlisting}
	

Der Konstruktor wird definiert, wir erfassen die ID, die Position und die Textur. 
\begin{lstlisting}
	...
	Tile(const unsigned int& ID, sf::IntRect* TextureRectangle, const sf::Sprite &Sprite, const sf::Vector2f& TilePosition)
	...
\end{lstlisting}

Die die Member werden mit den Parametern initialsiert 
\begin{lstlisting}
	...
	: tile_ID(ID), tile_texRect(TextureRectangle), tile_sprite(Sprite), tile_position(TilePosition)
	...
\end{lstlisting}

Der Textur wird die Position und Texture Rectangle zugewiesen. 
\begin{lstlisting}
	...
	{
		tile_sprite.setPosition(tile_position);
		tile_sprite.setTextureRect(*tile_texRect); 
	}
	...
\end{lstlisting}

Alle Teile zusammen ergeben einen Funktionierenden Datentyp in dem wir alle notwendigen Information Speichern. 
\begin{lstlisting}
	struct Tile
	{
		Tile(const unsigned int& ID, sf::IntRect* TextureRectangle, const sf::Sprite &Sprite, const sf::Vector2f& TilePosition)
		: tile_ID(ID), tile_texRect(TextureRectangle), tile_sprite(Sprite), tile_position(TilePosition) 
		{
			tile_sprite.setPosition(tile_position);
			tile_sprite.setTextureRect(*tile_texRect); 
		}
		...
	};
\end{lstlisting}

Doch noch ein Schritt fehlt um mit der Generierung zu Starten. Da die Karte in der Klasse abgespeichert werden soll fügen wir den entsprechenden Member hinzu. 
Hiefür verwenden wir den Typ std::vector. Dieser fungiert als herkömmliche liste, die aber nicht statisch initalisiert werden muss wie es beispielsweise bei 
Tile arr[]; der Fall wäre. Später sollen andere Klassen noch auf die Karte zugreifen also schreiben wir sie als public. 
\begin{lstlisting}
	...
	std::vector<std::vector<Tile>> p_tileMap;
	...
\end{lstlisting}

\captionof{lstlisting}{Tile}

Nach dem alle vorbereitungen getroffen worden sind, können wir mit der Generierung beginnen. 

\begin{lstlisting}
	...
	void Generate();
	...
\end{lstlisting}

Zunächst wollen wir p\_tilemap mit Tiles füllen. Dafür iterieren wir entlang der x-achse und erstellen einen std::vector den wir mit den noch leeren Tiles füllen. Anchließend wird der gefüllte std::vector an p\_tilemap angefügt.

\begin{lstlisting}
void Generate()
{
	for (int i = 0; i < m_width; i++)
	{
		std::vector<Tile> tileMap_row;
		
		for (int j = 0; j < m_height; j++)
		{
			
			tileMap_row.push_back(tile);
		}
		
		p_tileMap.push_back(tileMap_row);
	}
}
\end{lstlisting}
An dieser Stelle verwenden wir Prozedurale Generation - Diese ist zu diesem Zeitpunkt nicht implementiert oder Erklärt. Daher machen wir uns eine Mentale Notiz und 
kommen später wieder zurück. Die grundsätzliche Idee ist es eine 2-Demensionale Liste mit Semi-Zufälligen Werten zu füllen. Jeder dieser Werte stellt ein Bestimmtes Biom da. Demnach laden wir die textur welche dem Biom entspricht. Vorerst schreiben wir also: 
\begin{lstlisting}
...
int biome = // Das Biom an der stelle [i][j]
...
\end{lstlisting}

Wir verwenden das Biom um die richtige Textur, sowie die richtige Position zu finden. 
\begin{lstlisting}
...
sf::IntRect texRect(m_tileSize.x * biome, 
		           m_tileSize.y * biome, 
			   m_tileSize.x, m_tileSize.y);
sf::Vector2f tilePos(m_tileSize.x * i, m_tileSize.y * j);
...
\end{lstlisting}

Anschließend verwenden wir den Konstrukter des Tile-Struct um dem Tile die Daten zuzuweisen. 

\begin{lstlisting}
	...
	Tile tile(index, biome, &texRect, m_tilesprite, tilePos);
	...
\end{lstlisting}

Die vollständige Methode: 

\begin{lstlisting}
void Generate()
{
 for (int i = 0; i < m_width; i++)
 {
  std::vector<Tile> tileMap_row;
  for (int j = 0; j < m_height; j++)
  {
   int biome = // Das Biom an der stelle index
			
   sf::IntRect texRect(m_tileSize.x * biome, 
   			      m_tileSize.y * biome, 
   			      m_tileSize.x, m_tileSize.y);
   					   
   sf::Vector2f tilePos(m_tileSize.x * i, m_tileSize.y * j);
			
   Tile tile(index, biome, &texRect, m_tilesprite, tilePos);
			
   tileMap_row.push_back(tile);
  }
		
  p_tileMap.push_back(tileMap_row);
 }
}
\end{lstlisting}

\captionof{lstlisting}{Map::Generate()}

Nachdem wir Initalize() und Generate() bereits erstellt haben, fehlt uns nur noch Draw(). 

\begin{lstlisting}
	...
	void Draw();
	...
\end{lstlisting}

Vorerst hat die Methode nur die Aufgabe alle Texturen der Tiles zu zeichnen. Das problem ist hierbei aber das wir gleich die Ganze Map rendern, das wiederrum führt zu problemen mit der Performance. Es sollte also nur ein Teil bzw. "Chunk" der Map gerendert werden. 

\begin{lstlisting}
	void Draw(sf::RenderWindow& Window)
	{
		for(auto& i : p_tilemap){
			for(auto& j : i){
				window.draw(j.tile_sprite); 
			}
		}	
	}
\end{lstlisting}

Die Map-Klasse ist Fertig. (Vorerst)
Auf der TO-DO-List: Prozedurale Generation mit Perlin Noise und Chunk rendering

\captionof{lstlisting}{Map::Draw()}

\newpage
\section{Der Spieler}

Erste uebersicht der Klasse: 
\begin{lstlisting}
class Player
{
 public:
	void Initalize();
	void Update();
	void Draw();
	
 private:
	void MovePlayer();
};
\end{lstlisting}

Wir beginnen mit der Methode Initlize(), zweck der Methode sollte zu diesem Punkt schon klar sein. Wir stellen uns also die Frage welche Daten die Klasse beeinhalten Soll. 
fuer dieses Projekt werden wir folgende verwenden: Geschwindigkeit, Lebenspunkte, Position, die Hitbox* und die Textur. als public schreiben wir die Hitbox, Position und Lebenspunkte. Der Rest wird als privat geschrieben. 

\begin{lstlisting}
	...
	void Initalize();
	...
\end{lstlisting}

\begin{lstlisting}
	...
public:
	sf::RectangleShape p_hitbox;
	int				   p_health;
private:
	sf::Sprite		   m_sprite;
	sf::Vector2f	   m_position;
	float			   m_speed;
	int movementIndicator = 0; // wird spaeter erklaert 
	
\end{lstlisting}

Nun fuegen wir die notwendigen hinzu und initalisieren die Member. 

\begin{lstlisting}
...
void Initalize(Textureholder& Textures, const float& Speed, const int& Health, const sf::Vector2f& Position);
...
\end{lstlisting}

Das für dieses Projekt verwendete Spieler-Sprite hat eine größe von 64x64 daher setzten wir in sf::IntRect die breite sowie die höhe auf 64 pixel. Die hitbox soll der Spritegröße entsprechen und bekommt daher den selben wert zugewiesen. Wir verwenden den Textureholder* um auf die Spieler Textur zuzugreifen. 

\begin{lstlisting}
	...
	m_sprite.setTexture(Textures.Get(Textures::ID::Player));
	m_sprite.setTextureRect(sf::IntRect(0, 0, 64, 64));
	m_sprite.setPosition(Position);
	
	p_hitbox.setSize(sf::Vector2f(64, 64));
	...
\end{lstlisting}

Die Update Methode der Spieler klasse soll die Spielerbewegungen festhalten, daher verändern wir die Position um die Geschwindigkeit * Deltatime gleichzeitg verschieben wir den Ausschnitt der textur (TextureRectangle). Fue diesen Zweck definieren wir die Makros* UPWARDS, DOWNWARDS, LEFTWARDS und RIGHTWARDS. die Werte stellen die Position in der jeweiligen Sprites innerhalb der Textur da. Um festzustellen ob die Spielerposition sich mit einem Hinderniss überschneidet übergeben wir die Map-Klasse an die Methode. Zusätzlich passen wir die Hitbox an die positon des Spielers an. 


\begin{lstlisting}
	...
	#define FORWARD 0
	#define LEFTWARD 1
	#define BACKWARD 2
	#define RIGHTWARD 3
	...
\end{lstlisting}

\begin{lstlisting}
	...
	void Update(const float& Deltatime, MapGenerator &Map)
	{
		MovePlayer(Deltatime, Map);
		p_hitbox.setPosition(m_position);
	}
	...
\end{lstlisting}

Definieren wir die MovePlayer Methode. 
\begin{lstlisting}
	...
	private: 
	 void MovePlayer(const float &Deltatime, MapGenerator &Map); 
	...
\end{lstlisting}

Fuer jede der Richtung verwenden die von SFML bereitsgestellte Methode " sf::KeyBoard::isKeyPressed ". Diese registriert eingaben die über die Tastatur getaetigt werden. 

\begin{lstlisting}
	...
	if (sf::Keyboard::isKeyPressed(sf::Keyboard::W))
	...
\end{lstlisting}

wir legen die neue position fest. Da der Spieler in richtung oben laufen soll multiplplizieren wir die änderung an der y-achse mit -1. Das funktioniert weil der 
punkt (0,0) Oben-Links befindet. (Quelle)  

\begin{lstlisting}
	...
	sf::Vector2f newposition = m_position + sf::Vector2f(0, -1) * m_speed * dt; 
	...
\end{lstlisting}

An dieser stelle ist anzumerken das es wohl bessere wege gaebe um das problem zu lösen. Da der Userinput bei jeder iteration der Update methode abgefragt wird, kommt es zu mehreren hunderten abfragen pro sekunde. Da das sich nicht mit der Anfordung ein visuell stimulierendes Spiel zu programmieren deckt, verhindern wir dies in dem wir eine 
Membervariabel deklarieren welche bei jeder Iteration mitlaeuft. Die Werte 9 und 5 sind hier trivial.  
\newline 
Zuletzt weisen wir den Member m\_position die neue position zu.  
\begin{lstlisting}
	...
	m_sprite.setPosition(newposition);

	m_sprite.setTextureRect(sf::IntRect((movementIndicator / 5) * SPRITEUNIT, FORWARD, SPRITEUNIT, SPRITEUNIT));
	movementIndicator++;
	if (movementIndicator / MOVEMENT == 9) {
		movementIndicator = 0;
	}
	

	m_position = m_sprite.getPosition();
}
...
\end{lstlisting}

Da wir verhindert müssen das der Spieler mit hindernissen kollidert ueberpruefen wir ob die Spielerpostion auf der Tilemap bereits belegt ist. Dauer definieren wir die
methode YouShallPass.

\begin{lstlisting}
	...
	bool YouShallPass(const sf::Vector2f& reisepass, MapGenerator &map) {
		int x = reisepass.x / map.GetTileSize().x, y = reisepass.y / map.GetTileSize().y; 
		if (map.p_tileMap[x][y].occupied != true) {
			return true; 
		}
		return false; 
	}
	...
\end{lstlisting}

Da mit der jetzigen programmierung der Spieler daran gehindert wäre durch gegner durchzulaufen schreiben wir uns noch die Hilfsfunktion "IsEnemy" diese überprüft ob sich die positon mit einem gegner überschneiden würde. Wir verwenden wir den Member des Tile-Structs welche wir im Kapitel 3.3 bereits kennengelernt haben, occupationID. 

\begin{lstlisting}
	...
	bool IsEnemy(MapGenerator& map, const int& x, const int& y) {
		Textures::ID type = map.p_tileMap[x][y].occupationID; 
		switch (type)
		{
			case Textures::ID::Zombie:
			return true; 
		}
		
		return false; 
	}
	...
\end{lstlisting}

Anschließend fuegen wir der Abfrage in der YouShallPass Methode noch die IsEnemy option hinzu.

\begin{lstlisting}
	...
    if (map.p_tileMap[x][y].occupied != true || IsEnemy(map, x,y)) 
	...
\end{lstlisting}

Der Code-Block wird zusättzlich noch für die fälle S, A und D bzw. (0, 1), (-1, 0), (1, 0) wiederhohlt.  

\begin{lstlisting}
	...
	void MovePlayer(const float& dt, MapGenerator& map)
	{
		if (sf::Keyboard::isKeyPressed(sf::Keyboard::W)) {
			sf::Vector2f newposition = m_position + sf::Vector2f(0, -1) * m_speed * dt; 
			
			if (YouShallPass(newposition, map)) {
				m_sprite.setPosition(newposition);
				m_sprite.setTextureRect(sf::IntRect((movementIndicator / MOVEMENT) * SPRITEUNIT, FORWARD, SPRITEUNIT, SPRITEUNIT));
				movementIndicator++;
				if (movementIndicator / MOVEMENT == 9) {
					movementIndicator = 0;
				}
			}
			
		}
		...
		
		m_position = m_sprite.getPosition();
	}
	...
\end{lstlisting}



\newpage
\section{Perlin Noise}


\begin{lstlisting}
DOOOOOD
\end{lstlisting}








