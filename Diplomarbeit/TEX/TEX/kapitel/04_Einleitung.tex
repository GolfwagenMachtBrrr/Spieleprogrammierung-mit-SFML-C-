In der heutigen digitalen Ära spielt die Entwicklung von Anwendungssoftware eine immer wichtigere Rolle, insbesondere im Kontext der zunehmenden Digitalisierung verschiedener Aspekte des täglichen Lebens.


\section{Bestehende Software}
Apps können durch die Verwendung verschiedener Programmiersprachen wie Java, Kotlin für Android oder Swift für iOS entwickelt werden. Die Entwicklungsumgebung bietet Werkzeuge wie Android Studio für Android oder Xcode für iOS, die Entwicklern helfen, Benutzeroberflächen zu gestalten, Funktionalitäten zu implementieren und mit APIs zu interagieren. Anschließend wird die App auf einem Emulator oder einem physischen Gerät getestet und schließlich in den jeweiligen App Stores veröffentlicht. Selbst Alan Turing (vgl.~\cite{Turing1936}) konnte das nicht ahnen.


\subsection{Android}
Android ist ein Betriebssystem für mobile Geräte, das von Google entwickelt wird (siehe \cite{android}). Es basiert auf dem Linux-Kernel und bietet eine offene Plattform für Entwickler. Android ermöglicht die Entwicklung vielfältiger Anwendungen, von Spielen über Produktivitäts-Apps bis hin zu sozialen Medien. Der Google Play Store bietet eine riesige Auswahl an Apps für Android-Geräte.

